%\usepackage{floatrow}
\usepackage[outercaption]{sidecap}    
\sidecaptionvpos{figure}{b}
%%%%%%%%%%%%%%%%%%%%%
\usepackage{pdftexcmds}
\makeatletter
\newcommand{\stringcases}[3]{%
	\romannumeral
	\str@case{#1}#2{#1}{#3}\q@stop
}
\newcommand{\str@case}[3]{%
	\ifnum\pdf@strcmp{\unexpanded{#1}}{\unexpanded{#2}}=\z@
	\expandafter\@firstoftwo
	\else
	\expandafter\@secondoftwo
	\fi
	{\str@case@end{#3}}
	{\str@case{#1}}%
}
\newcommand{\str@case@end}{}
\long\def\str@case@end#1#2\q@stop{\z@#1}
\makeatother

%%%%%%%%%%%%%%%%5


%\newcommand*{\dothis}[1]{%
%	\stringcases
%	{#1}%
%	{%
%		{a}{so you typed a}%
%		{b}{now this is b}%
%		{c}{you want me to do c?}%
%	}%
%	{[nada]}%
%}


\usepackage{siunitx}
\usepackage{catchfile}

\newcommand*\readNumFromFile[2][\num]{%
	\CatchFileDef{\tempnum}{#2}{}%
	#1{\tempnum}%
}



\usepackage{booktabs,longtable}
\usepackage{adjustbox}
\usepackage{csvsimple} % For csv importing.

\makeatletter
\csvset{
	autotabularcenter/.style={
	file=#1,
	after head=\csv@pretable\begin{tabular}{|*{\csv@columncount}{c|}}\csv@tablehead,
	table head=\hline\csvlinetotablerow\\\hline,
	late after line=\\,
	table foot=\\\hline,
	late after last line=\csv@tablefoot\end{tabular}\csv@posttable,
	command=\csvlinetotablerow},
}
\makeatother
\makeatletter
\csvset{
	autotabularcentercolorheader/.style={
		file=#1,
		after head=\csv@pretable\begin{tabular}{|*{\csv@columncount}{c|}}\csv@tablehead,
			table head=\hline\rowcolor{headerColor}\csvlinetotablerow\\\hline,
			late after line=\\,
			table foot=\\\hline,
			late after last line=\csv@tablefoot\end{tabular}\csv@posttable,
		command=\csvlinetotablerow},
}
\makeatother

\newcommand{\csvautotabularcenter}[2][]{\csvloop{autotabularcenter={#2},#1}}
\newcommand{\csvstyleread}[3][]{\csvloop{#2={#3},#1}}

%\newcommand*{\readcsv}[3][]{%
%	\stringcases{#1}{
%		{a}{\csvloop{autotabularcenter={#3},#2}}%
%	}%
%	{[nada]}
%}
\newcommand*{\readcsv}[2]{%
	\stringcases
	{#1}%
	{%
		{default}{\readfromcsvs{autotabularcenter}{#2}}%
		{simplecenter}{\csvstyleread{autotabularcenter}{#2}}%	
		{centercolorheader}{\csvstyleread{autotabularcentercolorheader}{#2}}%
		{b}{now this is b}%
		{c}{you want me to do c?}%
	}%
	{[nada]}%
}


\usepackage{color}
\usepackage[table, svgnames, dvipsnames]{xcolor}
\usepackage{listings}
\usepackage{tcolorbox}
\tcbuselibrary{breakable}
\tcbuselibrary{skins}
\tcbuselibrary{listings}

\definecolor{dkgreen}{rgb}{0,0.6,0}
\definecolor{gray}{rgb}{0.5,0.5,0.5}
\definecolor{lightgray}{rgb}{0.8,0.8,0.8}
\definecolor{mauve}{rgb}{0.58,0,0.82}

\lstset{frame=tBLr,
	%	frameround=fttt,
	language=Python,
	aboveskip=3mm,
	belowskip=3mm,
	showstringspaces=false,
	columns=flexible,
	basicstyle={\small\ttfamily},
	lineskip=0.5em,
	xleftmargin=23pt,
	xrightmargin=9pt,
	framexleftmargin=17pt,
	framexrightmargin=5pt,
	framexbottommargin=2pt,
	numbers=left,
	firstnumber=1,
	numberstyle=\tiny\color{gray},
	keywordstyle=\color{blue},
	commentstyle=\color{dkgreen},
	stringstyle=\color{mauve},
	rulecolor=\color{black},
	breaklines=true,
	breakatwhitespace=true,
	tabsize=4
}



\definecolor{shadecolor}{gray}{0.95}
\definecolor{captionbox}{cmyk}{0.43, 0.35, 0.35,0.01}


\lstdefinelanguage{text}{
	keywords={},
}
\newtcblisting[auto counter]{wgetlisting}[2][]{%
	listing only,
	breakable,
	top=0.5pt,
	bottom=0.5pt,
	colback=red!5!white,
	colframe=red!25,
	left=6mm,
	sharp corners,
	boxrule=0pt,
	bottomrule=1pt,
	toprule=1pt,
	enhanced jigsaw,
	listing options={%style=tcblatex,
		frame=none,
		language=text,
		xleftmargin=1pt,
		xrightmargin=1pt,
		aboveskip=0mm,
		belowskip=0mm,
		lineskip=-0.15em,
		framexbottommargin=0pt,
		%
		numbers=left,
		numberstyle=\tiny\color{red!75!black},
		moredelim={[is][keywordstyle]{@@}{@@}},
		basicstyle=\normalsize\ttfamily,
		breaklines=true,
		breakautoindent=false,
		breakindent=0pt,
		escapeinside={(*}{*)},
	},%
	lefttitle=0pt,
	coltitle=black,
	colbacktitle=white,
	title={Listing \thetcbcounter:  #2},#1%  
	borderline north={1pt}{14.4pt}{red!25,dashed},
}
% 

%https://tex.stackexchange.com/questions/476100/lstlisting-line-number-gaps
\makeatletter
\let\orig@lstnumber=\thelstnumber
\newcommand\lstsetnumber[1]{\gdef\thelstnumber{#1}}
\newcommand\lstresetnumber{\global\let\thelstnumber=\orig@lstnumber}
\newcommand\bashnumbering{\lstsetnumber{\texttt{>>}}}
\makeatother



