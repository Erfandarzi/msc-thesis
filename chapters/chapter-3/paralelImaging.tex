\section{تصویر برداری موازی}

در این قسمت برای سرعت بخشیدن به فرایند تصویربرداری \mri روش هایی معرفی می‌شود که به تصویر برداری موازی 
\LTRfootnote{Parallel Imaging}
موسوم هستند.
در تصویربرداری \mri دادگان مستقیما از تصویر برداشته نمی‌شوند بلکه از فضای فوریه ‌ی تصویر موسوم به \kspace اخذ می‌شوند. (شکل \ref{fig:kspace-fov-res})






\begin{figure}[t!]
	\centering
	\begin{RTLcopyrightBox}{\linewidth}{\doiSource{10.1002/jmri.23639}}
		\subfigure[میدان دید کامل و رزولوشن کامل]{
			\includegraphics[width=0.3\linewidth]{chapters/chapter-3/figs/kspace-full-fov-full-res}
			\label{subfig:kspace-full-fov-full-res}}
		\hfill
		\subfigure[میدان دید کامل و رزولوشن کمتر]{
			\includegraphics[width=0.3\linewidth]{chapters/chapter-3/figs/kspace-full-fov-lower-res}
			\label{subfig:kspace-full-fov-lower-res}}
		\hfill
		\subfigure[میدان دید کمتر و رزولوشن کامل]{
			\includegraphics[width=0.3\linewidth]{chapters/chapter-3/figs/kspace-smaller-fov-full-res}
			\label{subfig:kspace-smaller-fov-full-res}}
		\hfill
	\end{RTLcopyrightBox}
	\removevspace
	\caption{}
	\label{fig:kspace-fov-res}
\end{figure}







سیگنال های بدست آمده عموما در جهت $k_x$ به صورت فرکانسی کد می‌شوند و در جهت $k_y$ به صورت خطوطی از کد کردن فاز می‌باشند. همچنین اگر تصویر برداری سه بعدی مدنظر باشد یک کد کردن دوم فاز نیز در جهت $k_z$ صورت می‌گیرد. زمان کل اخذ داده
\LTRfootnote{Total acqusition time}($T_A$)
برای یک اسکن دو بعدی به صورت زیر می‌باشد.

\removevspace
\begin{equation}
T_A = T_R \times N_{PE}
\end{equation}

که $T_R$ زمان تکرار
\LTRfootnote{Repetition time}
یا زمانی که نیاز است تا یک خط از \kspace در طول $k_x$ دریافت شود می‌باشد. همچنین $N_{\mathrm{PE}}$ نیز تعداد خطوط کد کردن فاز در جهت $k_y$ می‌باشد. در یک تصویر برداری سه بعدی، تعداد کد کردن پارتیشن ها
\LTRfootnote{Partition encoding}
($N_{\mathrm{PART}}$)
نیز باید به حاصل ضرب اضافه کرد.

$T_R$
کمک می‌کند که کنتراست تصویر را تنظیم کنیم و $N_{\mathrm{PE}}$
روزولوشن تصویر در جهت کدکردن فاز را تعیین می‌کند.

برای کاهش زمان استخراج داده، یا باید داده های \kspace سریع‌تر استخراج شود(که به معنی کاهش زمان $T_R$
است) و یا تعداد داده کمتری استخراج شود (که به معنای کاهش $N_\mathrm{PE}$ می‌باشد).


سرعت جمع آوری داده های \kspace، کنتراست مطلوب تصویر را تنظیم می‌کند. برای بعضی از انواع اسکن مانند اسپین اکو
\LTRfootnote{Spin Echo}
، $T_R$ باید به اندازه ای باشد تا کنتراست مطلوب تصویر تولید شود. در باقی انواع اسکن مانند 
گرادیان
\LTRfootnote{Spoiled Gradient Echo}
یا سری‌های تقدیمی آزاد حالت ماندگار متعادل
\LTRfootnote{Balanced Steady-State Free-Precession Sequences}
، کاهش زمان $T_R$ با حفظ کنتراست تصویر ممکن است.

اما یک محدودیت فیزیکی نیز وجود دارد. قطع و وصل کردن سریع گرادیان با میدان قوی می‌تواند یک جریان الکتریکی در بدن بیمار القا کند که آن نیز می‌تواند بالقوه باعث ایجاد تحریکات عصبی حاشیه‌ای
\LTRfootnote{Peripheral Nerve Stimulation}
 شود\cite{ParallelMRImaging2012}.

دیدگاه دیگر برای کاهش زمان $T_A$، کاهش تعداد داده های جمع آوری شده می‌باشد. یک راه برای این کاهش، کاهش ساده‌ی $k_{y, \max}$
می‌باشد در حالی که فاصله‌ی $\Delta k_y$ حفظ شده است (شکل \ref{subfig:kspace-full-fov-lower-res}).
از آن جا که طبق رابطه \ref{eq:fov-res-kspace}
رزولوشن در جهت $y$ با معکوس $k_{y,\max}$ متناسب است، این کار باعث کاهش رزولوشن تصویر و مات شدن آن می‌شود.
اگر رزولوشن تصویر بخاطر مسایل کلینیکالی بخواهد ثابت بماند، یک راه آن است که مانند شکل \ref{subfig:kspace-smaller-fov-full-res} برخی از خطوط کدکردن فاز حذف شوند. با این کار $\Delta k_y$ زیاد می‌شود و مطابق رابطه‌ی \ref{eq:fov-res-kspace} باعث کاهش در میدان دید تصویر می‌شود که درصورتی که ابعاد شئ از ابعاد میدان دید بزرگتر باشد، می‌تواند باعث ایجاد اختلاط مکانی
\LTRfootnote{Spatial Aliasing}
می‌شود. در حقیقت \lr{FOV} (که توسط فاصله‌ی خطوط کد کردن فاز تعیین می‌شود)، باید حداقل به بزرگی ابعاد شئ مورد تصویر برداری باشد. این الزام بر روی \lr{FOV} و محدوده‌ی نمونه برداری \kspace به عنوان نرخ نایکویست
\LTRfootnote{Nyquist criterion}
شناخته می‌شود. اگر نرخ نایکوییست در هردو جهت $k_x$ و  $k_y$ رعایت شود، تصویر می‌تواند مانند شکل \ref{subfig:kspace-full-fov-full-res}
بدون اختلاط مکانی بازسازی شود\cite{ParallelMRImaging2012}.



\subsection{روش های تصویربرداری موازی}

هنگامی که سرعت بخشیدن به جمع آوری داده های \mri با جمع آوری تعداد کمتری خطوط کدکردن فاز ممکن است که قبل از استفاده تصاویر در کاربرد های کلینیکی، اختلاط مکانی آن ها حذف شود. روش های تصویربرداری موازی در راستای حل کردن این مشکل، ایجاد شده اند. همه روش های تصویربرداری موازی سه مشخصه‌ی زیر را مشترک دارند\cite{ParallelMRImaging2012}:


\begin{figure}[t!]
	\centering
	\subfigure[]{
		\copyrightbox[b]{
			\includegraphics[width=0.45\linewidth]{chapters/chapter-3/figs/coil-sensivity-2}
			\label{subfig:coil-sensivity-2}
		}{\doiSource{10.1017/CBO9780511545405}}}
	\hfill
	\subfigure[]{
		\copyrightbox[b]{
			\includegraphics[width=0.36\linewidth]{chapters/chapter-3/figs/coil-sensivity-8}
			\label{subfig:coil-sensivity-8}
		}{\doiSource{10.1002/jmri.23639}}}
	\removevspace[1]
	\caption{}
	\label{fig:coil-sensivity}
\end{figure}



\removevspace[1]
\begin{enumerate}
	\item
	داده های \kspace در جهت کدکردن فاز برای کاهش زمان اسکن، زیر نرخ نمونه برداری شده اند (در تصویربرداری های سه بعدی می‌تواند در جهت کدکردن پارتیشن نیز چنین باشد).
	\textbf{فاکتور شتاب}
	\LTRfootnote{Acceleration factor}
	یا\textbf{ فاکتور کاهش}
	\LTRfootnote{Reduction factor}
	$R$ 
	به صورت نسبت مقدار دیتای \kspace برای نمونه برداری کامل تصویر به مقدار جمع آوری شده در یک استخراج تسریع یافته
	\LTRfootnote{Accelerated accqusition}
	، تعریف می‌شود. اگر نرخ نایکوییست رعایت نشود و میدان دید کمتر اندازه‌ی شئ باشد، تصویری دارای اختلاط را نتیجه می‌دهد.
	
	\item
	داده های بوسیله‌ی کانال‌های دریافت کننده ی مستقلی به جای یک سیم‌پیچ دریافت کننده همگن حجمی بزرگ (شکل \ref{subfig:coil-sensivity-8}) بدست می‌آیند. هریک از این سیم‌پیچ های دریافت کننده به حجمی که به آن‌ها نزدیک تر است حساس‌ترند.(شکل \ref{subfig:coil-sensivity-2})  که این یعنی که آن سیم‌پیچ ها یک منبع اضافه از اطلاعات مکانی جهت بازسازی تصویر را فراهم می‌کنند.

	\item
	یک الگوریتم مخصوص که به اطلاعات حساسیت سیم‌پیچ های مختلف برای بازسازی احتیاج دارد، برای ترکیب داده های زیر نرخ نمونه برداری شده از هر سیم‌پیچ دریافت کننده استفاده می‌شود تا یک تصویر نمونه برداری با میدان دید کامل بدون اختلاط بدست آید.
\end{enumerate}

چیزی که باید به آن توجه کرد این است که تصویربرداری موازی یک سری پالس خاص نیست.
تعداد کانال های دریافت کننده در آرایه ای از سیم پیچ ها، مقدار ماکسیمم فاکتور شتاب را محدود می‌کند.
\hl{به طور کلی، فاکتور شتاب نمی‌تواند از تعداد سیم‌پیچ های آرایه بیشتر شود. }
هرچند این پارامتر معمولا خیلی کمتر انتخاب می‌شود تا تصویری با کیفیت مطلوب تولید شود.

روش های مختلفی در تصویربرداری موازی وجود دارد و آن سه مشخصه‌ی مذکور بین‌شان مشترک است.این الگوریتم ها را می‌توان به دو دسته تقسیم کرد.
\begin{enuminline}
	\item آن‌هایی که با تصاویر دارای اختلاط  کار می‌کنند(مانند \lr{SENSE})	
	و 
	\item
	 آن‌هایی که داده های از دست رفته‌ی \kspace را بازسازی می‌کنند (مانند \lr{GRAPPA}).
\end{enuminline}

\subsection{آرایه های سیم‌پیچ های دریافت کننده}


قبل از بحث در مورد متود های مختلف تصویربرداری موازی مهم است که در مورد یک سخت افزار اساسی در تصویربرداری موازی یا به عبارت دیگر آرایه دریافت کننده چندکاناله
\LTRfootnote{multichannel receiver array}،
صحبت کرد.
حساسیت یک کانال دریافت کننده تکی در یک آرایه به سیگنال‌هایی که از یک منطقه مکانی خاص می‌آیند، محدود می‌شود که
در شکل \ref{fig:coil-sensivity}
قابل مشاهده است. این حساسیت معمولا به شئ درون سیم‌پیچ های دریافت کننده بستگی دارد و بنابراین از یک بیمار به بیمار دیگر می‌تواند تغییر کند. 

\begin{figure}[t!]
	\centering
	\copyrightbox[b]{
		\includegraphics[width=0.4\linewidth]{chapters/chapter-3/figs/coil-sensivity-direction}
	}{\doiSource{10.1002/jmri.23639}}
	\removevspace[1]
	\caption{}
	\label{fig:coil-sensivity-direction}
\end{figure}

پروفایل های حساسیت، میدان دید مطلوب را پوشش می‌دهد. هنگامی که یک اسکن با چندین سیم پیچ دریافت کننده انجام می‌شود، تصاویر نتیجه شده از هر سیم‌پیچ باید با یکدیگر ترکیب شوند. این کار می‌تواند با استفاده از \textbf{مجموع مربعات} (و یا سایر تکنیک هایی که یک سیگنال همگن را پس از ترکیب نتیجه می‌دهند) انجام داد.

آرایه های متداول در  کاربردهای کلینیکالی از 4 تا بیشتر از 32 کانال مستقل را شامل می‌شود و حساسیت های متنوعی را در دو یا سه بعد دارند. از آنجایی که تصویربرداری موازی به این تفاوت های حساسیت سیم‌پیچ ها تکیه دارد، سرعت بخشیدن صرفا در جهت تغییرات حساسیت می‌تواند اعمال شود. برای مثال اگر پنج سیم پیچ مانند شکل
 \ref{fig:coil-sensivity-direction} 
در یک خط قرار گیرند، صرفا در جهت خط آرایه(جهتی که حساسیتی بین سیم پیچ ها تغییر می‌کند)، سرعت بخشیدن می‌تواند اتفاق بیفتد. \cite{ParallelMRImaging2012}



شکل \ref{fig:coil-sensivity-product}
که چگونه یک حساسیست ناهمگن سیم‌پیچ می‌تواند یک تصویر \mr را تغییر دهد. به زبان ریاضی، هر نقطه در تصویر شئ در نقطه متناظرش در نقشه حساسیت
\LTRfootnote{Sensivity Map}
ضرب می‌شود. بنابراین نقطه‌ A و B در تصویر سمت راست در شکل \ref{fig:coil-sensivity-product}
به ترتیب با $A=C_A \times I_A$ و $B=C_B \times I_B$ برابر هستند. بنابراین تصویر هر سیم‌پیچ برابر با ضرب تصویر شئ در نقشه‌ی حساسیت آن سیم پیچ می‌باشد.


\begin{figure}[t!]
	\centering
	\copyrightbox[b]{
		\includegraphics[width=0.9\linewidth]{chapters/chapter-3/figs/coil-sensivity-product}
	}{\doiSource{10.1002/jmri.23639}}
	\removevspace[1]
	\caption{}
	\label{fig:coil-sensivity-product}
\end{figure}

\subsection{یک مثال ساده از \lr{pMRI}}

فرض کنید یک آرایه از $N_c = 2$ سیم‌پیچ دریافت کننده در اختیار داریم که به صورت خیلی ایده‌آل هریک از آن ها دقیقا نیمی از میدان دید \lr{FOV}
را پوشش می‌دهد. مطابق شکل \ref{subfig:pMRI-rec-c}
سیم‌پیچ اول صرفا نیمه‌ی بالایی شئ و سیم‌پیچ دوم نیمه پایینی شئ را شناسایی می‌کند. با این تکنیک می‌توان با ضریب کاهش 2، تعداد خطوط کدکردن فاز را کاهش داد. چرا که مطابق شکل \ref{subfig:pMRI-rec-b}، سیم پیچ اول سیگنال هایی که نیمه پایینی شئ می‌آیند را دریافت نمی‌کند و سیم‌پیچ دوم نیز سیگنال هایی که نیمه بالایی شئ می‌آیند را دریافت نمی‌کند. بنابراین می‌توان این دو نیمه را همزمان با یکدیگر اندازه‌گیری کرد که باعث می‌شود زمان لازم برای استخراج داده نصف شود. در گام آخر نیز به سادگی دو نیمه بدست آورده را در کنار یکدیگر قرار می‌دهیم و یک تصویر با رزولوشن و میدان دید کامل بدست می‌آوریم. در تصویر خروجی نسبت سیگنال به نویز با ضریب $\sqrt{R}$ و تعداد خطوط کدکردن فاز نیز با فاکتور کاهش $R$ کاهش می‌یابد.



\begin{figure}[t!]
	\centering
	\begin{RTLcopyrightBox}{\linewidth}{\doiSource{10.1002/jmri.23639}}
		\subfigure[]{
			\includegraphics[width=0.15\linewidth]{chapters/chapter-3/figs/pMRI-rec-a}
			\label{subfig:pMRI-rec-a}}
		\hfill
		\subfigure[]{
			\includegraphics[width=0.15\linewidth]{chapters/chapter-3/figs/pMRI-rec-b}
			\label{subfig:pMRI-rec-b}}
		\hfill
		\subfigure[]{
			\includegraphics[width=0.15\linewidth]{chapters/chapter-3/figs/pMRI-rec-c}
			\label{subfig:pMRI-rec-c}}
		\hfill
		\subfigure[]{
			\includegraphics[width=0.15\linewidth]{chapters/chapter-3/figs/pMRI-rec-d}
			\label{subfig:pMRI-rec-d}}
		\hfill
		\subfigure[]{
			\includegraphics[width=0.15\linewidth]{chapters/chapter-3/figs/pMRI-rec-e}
			\label{subfig:pMRI-rec-e}}
	\end{RTLcopyrightBox}
	\removevspace[2.5]
	\caption{مثال ساده ای از \lr{pMRI}}
	\label{fig:pMRI-rec}
\end{figure}

%\begin{figure}
%	\centering
%	\copyrightbox[b]{
%		\includegraphics[width=0.8\linewidth]{chapters/chapter-3/figs/PILS-rec}
%	}{\doiSource{10.1002/jmri.23639}}
%	\removevspace[1]
%	\caption{}
%	\label{fig:pils-rec}
%\end{figure}


این ساده سازی و ایده‌آل نگری یک جهان واقعی را بازتاب نمی‌دهد چرا که در واقعیت حساسیت‌های $C$ 
به صورت نرم
\LTRfootnote{Smooth}
بر روی میدان دید \lr{FOV}
تغییر می‌کنند و به صورت عادی بایکدیگر در برخی از محتوا اشتراک دارند. اما به‌هر حال این مثال ساده چند خصوصیت و الزامات مهم تصویربرداری موازی را نشان می‌دهد.
\cite{SMASH-SENSE-PILS-GRAPPA--2004}

\begin{enumerate}
	\item
	چندین سیم‌پیچ دریافت کننده با حساسیت های مختلف بر روی \lr{FOV}
باید وجود داشته باشند.
	\item
هر کدام از سیم‌پیچ ها بایستی مسیر خود را تهیه کنند.
\item
در بازسازی تصویربرداری موازی، اطلاعات دقیقی از حساسیت های سیم‌پیچ های مختلف لازم است.
\item
نسبت سیگنال به نویز در یک تصویربرداری موازی، حداقل با ضریب $\sqrt{R}$ کاهش می‌یابد.
\end{enumerate}

\FloatBarrier
\subsection{تصویربرداری موازی جزئی با حساسیت های محلی شده(\lr{PILS})}


\textit{تصویربرداری موازی جزئی با حساسیت های محلی شده}
\LTRfootnote{partially Parallel Imaging with Localized Sensitivities}(\lr{PILS})
که در سال 2000 معرفی شده است
\cite{PILS-Griswold2000}
، ملاحظات قبلی را به شرایط جهان واقعی گسترش می‌دهد.
مطابق شکل \ref{fig:pils-rec-image}
هر یک از سیم پیچ های دریافت کننده، حساسیتی کاملا محلی دارند بطوریکه در یک منطقه متمایز $Y_c$ و در باقی جاها صفر است و بنابراین هریک از آن ها یک منطقه‌ی متمایز از \lr{Fov} شئ را تحت پوشش قرار می‌دهند.




\begin{figure}[t!]
	\centering
	\copyrightbox[b]{
		\includegraphics[width=0.7\linewidth]{chapters/chapter-3/figs/PILS-rec-image}
	}{\doiSource{10.1097/01.rmr.0000136558.09801.dd}}
	\removevspace[1]
	\caption{روش \lr{PILS}}
	\label{fig:pils-rec-image}
\end{figure}

یک تصویربرداری موازی همراه با کاهش در جهت کدکردن فاز، تصاویر جزئی متناوبی را نتیجه می‌دهد. مادامی که \lr{FOV} کاهش‌یافته $Y_i$ بیشتر از منطقه‌ی محدود حساسیت انتخاب شود($Y_c < Y_i < Y$)، آن تصاویر جزئی به صورت کاملا جداگانه از یکدیگر در \lr{FOV} ظاهر می‌شوند.
به وسیله دانستن مکان دقیق هریک از سیم‌پیچ ها درون یک آرایه خطی، می‌توانیم تصویر جزئی متناظرش را در مکان درست استخراج کرد و در نهایت این زیر تصویر ها بایکدیگر ادغام می‌شوند و یک تصویر کامل با رزولوشن کامل را نتیجه می‌دهند.
\cite{SMASH-SENSE-PILS-GRAPPA--2004}


\lr{PILS}
محدود به ترکیب کافی سیم‌پیچ ها می‌شود به طوری که هرکدام از سیم پیج ها درون میدان دید \lr{FOV} و در جهت کدکردن فاز قرار داشته باشند.
\lr{PILS}
به \lr{SNR}
بهینه دست پیدا می‌کند که یعنی \lr{SNR}
صرفا با ضریب مورد انتظار $\sqrt{R}$ کاهش می‌یابد و \lr{SNR} تلفات دیگری ندارد چرا که تصویر نهایی از ورژن های شیفت یافته ی تصاویری با \lr{FOV}
کاهش یافته ساخته شده است. این واضح است که \lr{PILS}
شدیدا به سیم‌پیچ هایی با حساسیت های محلی احتیاج دارد در بخش بعدی با روش های عمومی‌تر آشنا می‌شویم.










\FloatBarrier
\subsection{کدکردن حساسیت (\lr{SENSE})}

روش \textit{کدکردن حساسیت}
\LTRfootnote{SENSivity Encoding}(\lr{SENSE})
در سال 1999 معرفی شد\cite{SENSE-1999} و یکی از روش هایی است که در کاربرد های تجاری مورد استفاده قرار می‌گیرد. 

\begin{figure}
	\centering
	\copyrightbox[b]{
		\includegraphics[width=0.7\linewidth]{chapters/chapter-3/figs/SENSE-rec-image}
	}{\doiSource{10.1097/01.rmr.0000136558.09801.dd}}
	\removevspace[1]
	\caption{روش \lr{SENSE}}
	\label{fig:sense-rec-image}
\end{figure}


روش باز سازی \lr{SENSE}
را می‌توان به‌طور خلاصه به عنوان یک الگوریتم \textit{بازکردن}
\LTRfootnote{Unfolding}
در حوزه‌ی تصویر، دسته بندی کرد.
\cite{SMASH-SENSE-PILS-GRAPPA--2004}
به شکل \ref{fig:sense-rec-image}
دقت کنید. 
 تصویر ایجاد شده در هر یک سیم‌پیچ ها را می‌توان به صورت مجموع حاصل ضرب چگالی پروتون (تصویر حقیقی) در میزان حساسیت سیم‌پیچ در آن ناحیه نوشت. به عبارت دقیق تر، معادله‌ی زیر برای تصویر کاهش یافته‌ی سیم پیچ $k$اُم برقرار است.

\removevspace
\begin{equation}\label{eq:I_k=sum_0^nA}
	\begin{aligned}
		I_k(x,y) = & C_k(x, y) \rho(x,y) + C_k(x, y + \Delta y) \rho(x, y + \Delta y) \\
				   & + \dots + C_k(x, y + n_A \Delta y) \rho(x, y + n_A \Delta y) \\
				 = & \sum_{n=0}^{n_A} C_k(x, y + n \Delta y) \rho(x, y + n \Delta y )
	\end{aligned}
\end{equation}

که در آن $n_A$ تعداد سیگنال های اختلاطی است که از مکانی به مکان دیگر متفاوت است و به اندازه های شئ مورد تصویر برداری، بستگی دارد. اگر ابعاد شئ دقیقا میدان دید بازسازی
 $\lrtext{FOV}_{\lrtext{rec}}$
 را پر کند، آنگاه $n_A = R$ در آن نقاط می‌شود و در باقی موارد همواره $n_A \le R$ می‌باشد. 
 همچنین $\Delta y$ نیز \textit{فاصله‌ی اختلاطی} 
\LTRfootnote{aliasing distance}
می‌باشد:

\removevspace
\begin{equation}
	\Delta y = \dfrac{\lrtext{FOV}_{\lrtext{rec}}}{R}
\end{equation}

اگر $N_c$ تا سیم‌پیچ داشته باشیم، 
معادله \ref{eq:I_k=sum_0^nA}
را به ازای $k$ های مختلف (از 1 تا $N_c$)
به صورت همزمان در یک ماتریس نوشت.


\begin{equation}\label{eq:matrix:I=Crho}
	\rescaleequation{0.85}{
	\left[\begin{array}{c}
		I_1(x, y) \\
		I_2(x, y) \\
		\vdots \\
		I_{N_c}(x, y)
	\end{array}\right]
	=\left[\begin{array}{cccc}
		C_{1}(x, y) & C_{1}(x, y+\Delta y) & \cdots & C_{1}\left(x, y+n_{A} \Delta y\right) \\
		C_{2}(x, y) & C_{2}(x, y+\Delta y) & \cdots & C_{2}\left(x, y+n_{A} \Delta y\right) \\
		\vdots & \vdots & \ddots & \vdots \\
		C_{N_c}(x, y) & C_{N_c}(x, y+\Delta y) & \cdots & C_{N_c}\left(x, y+n_{A} \Delta y\right)
	\end{array}\right]
	.
	\left[\begin{array}{c}
		\rho(x, y) \\
		\rho(x, y+\Delta y) \\
		\vdots \\
		\rho\left(x, y+n_{A} \Delta y\right)
	\end{array}\right]}
\end{equation}

و بنابراین معادله‌ی 
\ref{eq:matrix:I=Crho}
را می‌توان با نماد ساده شده زیر نوشت.
\cite{book:MRIfromPictureToProton}

\removevspace
\begin{equation}
	\vec{I}_{N_c\times1} = \mathbf{C}_{N_c \times n_A} . \vec{\rho}_{n_A\times1}
\end{equation}

حال به یک دستگاه معادلات خطی تبدیل شود. جواب بهینه برای این معادله را می‌توان با عملیات شبه‌وارون
\LTRfootnote{Pseudo Inverse}
پیدا کرد. همچنین باید توجه داشت که
$n_A \le R \le N_c$
برقرار است.(هرچند $R$ در این‌جا می‌تواند بیشتر از $N_c$ نیز شود تا جایی‌که $n_A$ با $N_c$ برابر شود.)

\removevspace
\begin{equation}\label{eq:rho=CdaggerI}
	\vec{\rho} = \mathbf{C}^\dagger . \vec{I} = (\mathbf{C}^H \mathbf{C})^{-1}\mathbf{C}^H . \vec{I}
\end{equation}


برای \textit{باز کردن}
تصویر اختلاطی کافیست که معادله‌ی \ref{eq:rho=CdaggerI}
را به ازای تمامی نقاط اجرا کنیم. بر خلاف روش \lr{PILS}
در روش \lr{SENSE}
برای تنظیمات دلخواه سیم‌پیچ ها می‌توان پردازش موازی را انجام داد اما هزینه آن کاهش اضافی \lr{SNR} است که به هندسه‌ی آرایه‌ای از سیم‌پیچ ها بستگی دارد. 
\cite{SMASH-SENSE-PILS-GRAPPA--2004}

\removevspace
\begin{equation}
	\lrtext{SNR}_{\lrtext{SENSE}} = \dfrac{\lrtext{SNR}_{\lrtext{full}}}{g.\sqrt{R}} 
\end{equation}

دلیل کاهش \lr{SNR} با ضریب $\sqrt{R}$ پیشتر توضیح داده شده است اما یک ضریب اضافی $g$ نیز وجود دارد که 
\textit{فاکتور هندسه}\LTRfootnote{geometry factor}
و یا \textit{فاکتور $g$}
\LTRfootnote{g-factor}
، نامیده می‌شود و همواره بیشتر و یا مساوی با 1 مقدار دارد. اگر حساسیت دو سیم‌پیچ دریافت کننده به شدت به یکدیگر همبسته باشند، تصاویر اختلاطی سخت تر از یکدیگر جدا شده و باعث کاهش \lr{SNR}
 می‌شود. هر چند این مشکل خودش را در معادله‌ی \ref{eq:rho=CdaggerI}
خود را نشان نداده است اما در کاهش \lr{SNR}
ظاهر می‌شود و از پیکسلی به پیکسل دیگر تغییر می‌کند و معمولا در مرکز تصویر مقادیر بیشتری را دارد چراکه در آنجا تعداد بیشتری از تصاویر همپوشانی دارند و حساسیت سیم‌پیچ های دریافت کننده شبیه به هم‌دیگر است.


مشکل اصلی بازسازی به روش \lr{SENSE}
این است که به حساسیت دقیق سیم‌پیچ ها احتیاج دارد. هرگونه خطایی در نقشه حساسیت سیم‌پیچ ها باعث ایجاد آرتیفکت هایی به شکل \textit{اختلاطِ باقی‌مانده} در تصویر بازسازی شده، می‌شود.
\cite{ParallelMRImaging2012}

پروفایل های حساسیت سیم پیچ های مختلف به مکان آن ها نسبت به آناتومی شئ مورد تصویربرداری  بستگی دارد و بنابراین نیاز به یک پیش اسکن دارد.
اگر بیمار حین عملیات تصویربرداری جابجا شود، حساسیت سیم پیچ ها ممکن است تغییر کند و نتایجی توأم با آرتیفک را نتیجه دهد.
این آرتیفکت ها را می‌توان با جمع آوری اطلاعات لازم برای محاسبه‌ی مجدد نقشه حساسیت سیم‌پیچ ها، تسکین داد. همچنین در نواحی ای که سیگنال ضعیف است مانند ریه ها و سینوس ها، تعیین حساسیت سیم‌پیچ ها به دلیل وجود نویز بالا در این مناطق می‌تواند سخت باشد.

اگرچه در تعیین نقشه حساسیت، چالش های بالقوه‌ای در روش \lr{SENSE}
وجود دارد اما \lr{SENSE}
و ورژن مختلف آن مانند \lr{mSENSE}
و یا \lr{ASSET}
به صورت روزانه در اسکن های تسریع یافته‌ی کلینیکی استفاده می‌شود.
\cite{ParallelMRImaging2012}

\FloatBarrier
\subsection{\lr{SMASH}}

از نظر تاریخی، 
\textit{جمع‌آوری داده های هارمونیک های مکانی به طور همزمان}
\LTRfootnote{SiMultaneous Acquisition of Spatial Harmonics}(\lr{SMASH})
، اولین تکنیک تصویربرداری موازی به حساب می‌آید(سال 1997)
\cite{SMASH-1997}.
عبارت زیر هارمونیک های مکانی که در واقع همان فرکانس های مکانی هستند، به این موضوع اشاره دارد که  این الگوریتم در حوزه‌ی \kspace کار می‌کند که بر خلاف دو روش \lr{SENSE}
و  \lr{PILS}
می‌باشد. این روش از ترکیب حساسیت سیم‌پیچ های مختلف استفاده می‌کند تا \textit{کد‌کردن فاز مجازی} بسازد.
برای این منظور ضرایب $n_k^{(m)}$ به گونه‌ای انتخاب می‌شوند تا ترکیبشان یک \textit{پروفایل مرکب حساسیت}
\LTRfootnote{Compsite sensivity profile}
 بسازد که به شکل یک سینوسی مرتبه $m$ در مکان تغییر کند یعنی:
 
 \removevspace
 \begin{equation}\label{eq:Ccomp}
 	C^\lrtext{comp}_m(x, y) =  \sum_{k=1}^{N_c}  n_k^{(m)} C_k(x,y) \cong e^{i2\pi m\Delta k_y y}
 \end{equation} 
 
 
 
\begin{figure}[t!]
	\centering
	\begin{RTLcopyrightBox}{\linewidth}{\doiSource{10.1017/9781107706958}}
	\subfigure[]{
		\includegraphics[width=0.25\linewidth]{chapters/chapter-3/figs/SMASH-rec-coils-0}
		\label{subfig:smash-rec-coils-0}}
	\hfill
	\subfigure[]{
		\includegraphics[width=0.25\linewidth]{chapters/chapter-3/figs/SMASH-rec-coils-1}
		\label{subfig:smash-rec-coils-1}}
	\hfill
	\subfigure[]{
		\includegraphics[width=0.4\linewidth]{chapters/chapter-3/figs/SMASH-rec-coils-pe}
		\label{subfig:smash-rec-coils-pe}}
	\end{RTLcopyrightBox}
	\removevspace[2.5]
	\caption{روش \lr{SMASH}}
\end{figure}
%	doi = {10.1017/9781107706958},




که در آن رابطه، $\Delta k_y = \frac{2\pi}{\lrtext{FOV}}$
می‌باشد و $k$ نیز بین 1 تا $N_c$ تغییر می‌کند.
حال برای سیگنال سیم پیچ $k$اُم را می‌نویسیم


\removevspace
\begin{equation}
	S_k(k_x, k_y) = \iint \rho(x,y) . C(x,y) . \exp(\frac{-t}{T_2^*}) . \exp(i2\pi x k_x) .\exp(i2\pi y k_y) . \mathrm{d}x \mathrm{d}y
\end{equation}

این رابطه مشابه رابطه‌ی \ref{eq:rel-dft-kspace}
است اما در آن جمله‌ی $C(x,y)$ نیز در نظر گرفته شده است. اگر این سیگنال ها را با ضرایب  $n_k^{(m)}$
با یکدیگر ترکیب کنیم، خواهیم داشت:


\removevspace
\begin{equation}
\begin{aligned}
	\sum_{k=1}^{N_c} n_k^{(m)} S_k(k_x, k_y) =&  \iint  \rho(x,y) . \left(n_k^{(m)}  \sum_{k=1}^{N_c} C(x,y) \right) . \exp(\frac{-t}{T_2^*}) \\
	&\qquad \exp(i2\pi x k_x) . \exp(i2\pi y k_y) . \mathrm{d}x \mathrm{d}y \\
	 =& \iint  \rho(x,y) . C_m^{\lrtext{COMP}} . \exp(\frac{-t}{T_2^*}) .\exp(i2\pi x k_x) . \exp(i2\pi y k_y) . \mathrm{d}x \mathrm{d}y 
\end{aligned}
\end{equation}


در صورتی که این رابطه را با رابطه ی \ref{eq:Ccomp}
ادغام کنیم، داریم:




\removevspace
\begin{equation}
	\begin{aligned}
		\sum_{k=1}^{N_c} n_k^{(m)} S_k(k_x, k_y) \cong& \iint  \rho(x,y) . \exp(\frac{-t}{T_2^*}) \\
		&\qquad e^{i2\pi x k_x} . e^{i2\pi y (k_y + m\Delta k_y)} . \mathrm{d}x \mathrm{d}y \\
		=&\ S^{\lrtext{COMP}}(k_x, k_y + m \Delta k_y)
	\end{aligned}
\end{equation}


در حقیقت در روش \lr{SMASH} نشان داده شده است که این ترکیب خطی از سیگنال های سیم‌پیچ‌های مختلف می‌تواند به وطر مجازی داده های \kspace 
را یک شیفت فرکانسی در جهت کدکردن فاز دهد و به شدت به تنظیمات سیم‌پیچ های مختلف وابسته است تا بتواند هارمونیک مکانی مطلوب را تولید کند.






\FloatBarrier
\subsection{\lr{GRAPPA}}

\LTRfootnote{Generalized Autocalibrating Partially Parallel Acquisitions}(\lr{GRAPPA})

\cite{GRAPPA-Griswold2002}








