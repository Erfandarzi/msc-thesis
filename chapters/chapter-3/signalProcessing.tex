\section{روش‌های متداول پردازش سیگنال}
در بخش \ref{ch:background} در مورد نحوه تصویر برداری \mri صحبت شد و دیدم که نمونه ها از فضای فوریه تصویر برداشته می‌شوند که آن فضا \kspace نام‌دارد. اما تصویر در نهایت یک تصویر حقیقی است و نه مختلط و موهومی، بنابراین می‌توان از این خاصیت استفاده کرد تا تعداد اخذ نمونه را به 50 درصد کاهش داد. البته در این روش باعث می‌شود که \lr{SNR} کم تر شود که در مورد آن نیز صحبت خواهد شد.

\begin{قضیه}[فوریه توابع حقیقی]
اگر یک تابع حقیقی $s(x,y)$ در احتیار داشته باشیم در این صورت داریم:
$$S(k_x, k_y) = \F[s(x,y)](k_x, k_y) \; \Rightarrow \; S(k_x, k_y) = S^*(-k_x,-k_y) $$
\end{قضیه}
























